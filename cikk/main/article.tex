\documentclass[12pt]{spieman}  % 12pt font required by SPIE;
%\documentclass[a4paper,12pt]{spieman}  % use this instead for A4 paper
\usepackage{amsmath,amsfonts,amssymb}
\usepackage{graphicx}
\usepackage{setspace}
\usepackage{tocloft}

\title{SPIE journal papers: sample manuscript showing style and formatting specifications}

\author[a,*]{G\'abor Galg\'oczi}
\author[a]{Second Author}
\author[b]{Third Author}
\author[a,b]{Fourth Author}
\affil[a]{University Name, Faculty Group, Department, Street Address, City, Country, Postal Code}
\affil[b]{Company Name, Street Address, City, Country, Postal Code}

\renewcommand{\cftdotsep}{\cftnodots}
\cftpagenumbersoff{figure}
\cftpagenumbersoff{table} 
\begin{document} 
\maketitle

\begin{abstract}
This document shows the required format and appearance of a manuscript prepared for SPIE journals. It is prepared using LaTeX2e with the class file \texttt{spieman.cls}. The abstract should consist of a single paragraph containing no more than 200 words. It should be a summary of the paper and not an introduction. Because the abstract may be used in abstracting and indexing databases, it should be self-contained (that is, no numerical references) and substantive in nature, presenting concisely the objectives, methodology used, results obtained, and their significance. A list of up to eight keywords should immediately follow, with the keywords separated by commas and ending with a period. The body of the manuscript should be double-spaced and fully justified.   
\end{abstract}

% Include a list of up to six keywords after the abstract
\keywords{optics, photonics, light, lasers, journal manuscripts, LaTeX template}

% Include email contact information for corresponding author
{\noindent \footnotesize\textbf{*}G\'abor Galg\'oczi,  \linkable{galgoczi.gabor@wigner.mta.hu} }

\begin{spacing}{2}   % use double spacing for rest of manuscript

\section{Introduction}
\label{sect:intro}  % \label{} allows reference to this section
This document shows the format and appearance of a manuscript prepared for submission to an SPIE journal. Note that this template is only intended to be used as a guideline for author convenience. It is designed for optimum clarity and ease of reading for editors and reviewers, but the template does not reflect the final page layout of a published journal paper. Accepted papers are professionally typeset in XML according to the layout and design of the journal. 

\section{Results}



\appendix    % this command starts appendixes

\section{Miscellaneous Formatting Details}
\label{sect:misc}


% \disclosures 
\subsection*{Disclosures}
Conflicts of interest should be declared under a separate header. If the authors have no relevant financial interests in the manuscript and no other potential conflicts of interest to disclose, a statement to this effect should also be included in the manuscript.

\acknowledgments 
This unnumbered section is used to identify those who have aided the authors in understanding or accomplishing the work presented and to acknowledge sources of funding.  

%%%%% References %%%%%

\bibliography{report}   % bibliography data in report.bib
\bibliographystyle{spiejour}   % makes bibtex use spiejour.bst

%%%%% Biographies of authors %%%%%

\vspace{2ex}\noindent\textbf{First Author} is an assistant professor at the University of Optical Engineering. He received his BS and MS degrees in physics from the University of Optics in 1985 and 1987, respectively, and his PhD degree in optics from the Institute of Technology in 1991.  He is the author of more than 50 journal papers and has written three book chapters. His current research interests include optical interconnects, holography, and optoelectronic systems. He is a member of SPIE.

\vspace{1ex}
\noindent Biographies and photographs of the other authors are not available.

\listoffigures
\listoftables

\end{spacing}
\end{document}