\documentclass[12pt]{spieman}  % 12pt font required by SPIE;
%\documentclass[a4paper,12pt]{spieman}  % use this instead for A4 paper
\usepackage{amsmath,amsfonts,amssymb}
\usepackage{graphicx}
\usepackage{setspace}
\usepackage{tocloft}

\title{A Geant4 based framework for the simulation of background of high-energy satellites caused by activation, cosmic, trapped and albedo particles}

\author[a,*]{G\'abor Galg\'oczi}
\author[a]{Second Author}
\author[b]{Third Author}
\author[a,b]{Fourth Author}
\affil[a]{University Name, Faculty Group, Department, Street Address, City, Country, Postal Code}
\affil[b]{Company Name, Street Address, City, Country, Postal Code}

\renewcommand{\cftdotsep}{\cftnodots}
\cftpagenumbersoff{figure}
\cftpagenumbersoff{table} 
\begin{document} 
\maketitle

\begin{abstract}
Abstract
\end{abstract}

% Include a list of up to six keywords after the abstract
\keywords{Geant4, GRB, gamma-rays, satellite, cosmic background}

% Include email contact information for corresponding author
{\noindent \footnotesize\textbf{*}G\'abor Galg\'oczi,  \linkable{galgoczi.gabor@wigner.mta.hu} }

\begin{spacing}{2}   % use double spacing for rest of manuscript

\section{Introduction}
\label{sect:intro}  % \label{} allows reference to this section

Particle background for satellites, mostly the ones aiming to investigate the high-energy Universe is considerable constraint. It is especially important for satellites without an anti-coincidence shield, e.g. for the more and more CubeSats which have recently been applied for scientific missions. 

A Geant4 based simulation framework is presented in this paper, which aims to aid future missions by providing a flexible, easy to use tool to quantify particle background. The simulation consists of two modules. The first determines the proton induced activation of the satellite. The second module quantifies the background signal which is induced by particles being present on the given orbit (e.g. cosmic, trapped and albedo particles). 

In order to keep the implementation of the simulation of a new satellite straightforward, the CAD model of the satellite is automatically read in by CADMesh []. The background of the HERMES and the CAMELOT CubeSat missions were determined. The input particle spectra (cosmic, trapped and albedo particles) for the background simulation of these missions were computed [JAKUB NE] by SPENVIS and several other models [mizuno].

The validation of the simulation framework was carried out by a set of dedicated experiments and simulations [Kento, the measurement of activation].

\section{The implementation of the framework} 
 
For the CAMELOT mission light propagataion was also included in the simulation in order to include the repsonse the scintillator as realisticly as possible. 
 
\subsection{Activation} 

\subsection{Proton-induced activation} 



[odaka paper], semi-analytic. This paper analytic. It was proven to work comparision with data taken from..
 
 
\newpage 
 
The differential flux is normed. Afterwards the normed integral flux is calculated for each bin. From 0 energy to the given energy.

$$ \int{1}{2} $$
 
A random number between zero and one is drawn. The bin that has

Check particle energy distirbution

cxb high energy

mizuno raised concern about cxb and gamma albedo??
 
\section{Methodology} 
 
The differential flux is normed. Afterwards the normed integral flux is calculated for each bin. From 0 energy to the given energy.

$$ \int{1}{2} $$
 
A random number between zero and one is drawn. The bin that has

Check particle energy distirbution

cxb high energy

mizuno raised concern about cxb and gamma albedo??
 
\section{Simulation of GRB induced signal in the detector} 

tbd: what is the background after polar orbits?
casing thickness needs to be optimized for:
more GRB signal
less CXB signal
less electrons...

\section{Results}



\appendix    % this command starts appendixes

\section{Miscellaneous Formatting Details}
\label{sect:misc}


% \disclosures 
\subsection*{Disclosures}


\acknowledgments 


%%%%% References %%%%%

\bibliography{report}   % bibliography data in report.bib
\bibliographystyle{spiejour}   % makes bibtex use spiejour.bst

%%%%% Biographies of authors %%%%%

\vspace{2ex}\noindent\textbf{First Author} is an assistant professor at the University of Optical Engineering. He received his BS and MS degrees in physics from the University of Optics in 1985 and 1987, respectively, and his PhD degree in optics from the Institute of Technology in 1991.  He is the author of more than 50 journal papers and has written three book chapters. His current research interests include optical interconnects, holography, and optoelectronic systems. He is a member of SPIE.

\vspace{1ex}
\noindent Biographies and photographs of the other authors are not available.

\listoffigures
\listoftables

\end{spacing}
\end{document}